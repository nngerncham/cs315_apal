\section{Skip Lists}

\subsection{Skip List vs. Ordered Map}

\subsubsection{Insert Latency}
\subsubsection{Search Latency}
\subsubsection{Delete Latency}

\subsection{Better Search Algorithm}

With the assumption that we can go up and down between each levels of the skip list with minor modification to the original, a more efficient search algorithm goes as follows.

\begin{algorithm}
\caption{Search algorithm for key $k$ in skip list $L$}
\begin{algorithmic}[1]
\Procedure{Search}{$k, L$}
	\State $V \gets$ Node with smallest key in $L$
	\While {$V_\text{next$\to$key} < k$} \Comment{Goes up to the highest level with max key less than $k$}
		\While {$V_\text{above} \neq \text{NULL} \cap V_\text{above$\to$next$\to$key} < k$}
			\State $V \gets V_\text{above}$
		\EndWhile
		\State $V \gets V_\text{next}$
	\EndWhile
	\While {$V_\text{below} \neq$ NULL} \Comment{Goes down to the bottom level}
		\State $V \gets V_\text{below}$
		\While {$V_\text{next$\to$key} < k$}
			\State $V \gets V_\text{next}$
		\EndWhile
	\EndWhile
	\If {$V_\text{key} = k$}
		\State \Return Success
	\Else
		\State \Return Failure
	\EndIf
\EndProcedure
\end{algorithmic}
\end{algorithm}

\subsubsection{Running Time Analysis}

\begin{tcolorbox}
\textbf{Lemma.} Given a key $k$, there are $O(\log d)$ valid levels with high probability where $d$ is the number of elements whose key is smaller $k$.
Here, a valid level refers to levels with at least one node $v$ where $v_\text{next$\to$key} < d$ or $v$ points to a node whose key is still less than $d$.

\textbf{Proof.} Given some key $k$, consider the element $e_k$ with key $k$. The probability of $e_k$ showing up in more than $c\log d$ levels is $1/2^{c\log d} = 1/d^c$.
As for the rest of the elements smaller than it, the probability of the remaining $d$ elements having more than $c\log d$ levels is bounded by $d/2^{c\log d} = 1/d^{c-1}$.
While the real height is bounded by $c\log n$ as shown in the lecture, a node in any level higher than $c\log d$ will not point to $e_k$ with high probability.
Therefore, there are $c\log d \in O(\log d)$ valid levels with high probability. $\quad\qedsymbol$
\end{tcolorbox}

Consider the while loop on lines 3-8. The path will keep going up (when $V\gets V_\text{above}$ on line 5) until there is no valid level left.
As proven in the lemma, there are $c\log d \in O(\log d)$ valid levels for some given key and $d$ elements with key less than $d$, so the number of times the path goes up is bounded by $c\log d$ for some constant $c$.
The expected number of times the path goes right (when $V \gets V_\text{next}$ on line 7) is $\frac{1}{1/2} = 2$ times so the number of times the path goes up is $2c\log d$.

Now, consider the second while loop on lines 9-14. In the worst case, the path can only go down (when $V \gets V_\text{below}$ on line 13) the same number of times it goes up which is $c\log d$ times.
Similarly, expected number of times the path goes right (when $V \gets V_\text{next}$ on line 11) is also $\frac{1}{1/2} = 2$ times so the number of times the path goes down is $2c\log d$.

In conclusion, the path will need to make $4c\log d \in O(\log d)$ steps to get from the node with the smallest key to the node with key $k$. Therefore, this search algorithm runs in $O(\log d)$ time.$\quad\qedsymbol$
