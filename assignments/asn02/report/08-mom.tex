\chapter{Median of Means}

First, we want to find the probability that a bad estimate or one that deviates from $\tau$ by at least $\epsilon$. Formally, we want to find $\pr{|T_i - \tau| \geq \epsilon}$. Recall that the mean $\ex{T_i} = \tau$ and the variance $\ex{(T_i - \tau)^2} = \beta$, we can bound this probability using Chebyshev's inequality to obtain
\[ \pr{|T_i - \tau| \geq \epsilon} \leq \frac{\beta}{\epsilon^2} \]

To find a number $K$ such that the median $\hat{T} := \text{median}(T_1, T_2, ..., T_K)$ is a good estimate of the true mean, we want more than half of our estimates $T_i$ to be good. This is because the median being a good estimate implies that the number of good estimates must be at least half since the median is the middle element of all estimates once sorted. If more than half estimates are off, then the middle one would be off as well.

Let $X_i$ be a random variable that is 1 if $T_i$ is a good estimate and 0 otherwise and let $X = X_1 + X_2 + \cdots + X_K$. Let us also bound the probability of having at least half of the estimates be good by some $\delta$. Since having more than half of the estimates be good with high probability implies that $\hat{T}$ is good with high probability, we have
\[ \pr{X \geq \frac{K}{2}} \geq 1 - \delta \iff \pr{|\hat{T} - \tau| \leq \epsilon} \geq 1 - \delta \]
This implies that bounding one will bound the other as well. Thus, we bound the probability on the LHS by applying Chernoff-Hoeffding on its complement. Let $\alpha = 1 - \epsilon^2/2\beta$.
\begin{equation*}
\begin{aligned}
	&& \pr{X \geq \frac{K}{2}} &\geq 1 - \delta \\
	\iff && \pr{X < \frac{K}{2}} &\leq \delta \\
	\iff && \pr{X < (1 - \alpha)K\frac{\beta}{\epsilon^2}} &\leq \delta = \exp\left(-\frac{\alpha^2}{2}K\frac{\beta}{\epsilon^2}\right)
\end{aligned}
\end{equation*}

Now, we simply solve for $K$ to find how many samples of means we need in order to use the median trick.
\[
\begin{aligned}
	\exp\left(-\frac{\alpha^2K\beta}{2\epsilon^2}\right) &= \delta \\
	\frac{\alpha^2K\beta}{2\epsilon^2} &= \ln\left(\frac{1}{\delta}\right) \\
	K &= \frac{2\epsilon^2}{\alpha^2\beta}\ln\left(\frac{1}{\delta}\right) \\
	K &= \frac{2\epsilon^2}{(1 - \frac{\epsilon^2}{2\beta})^2\beta}\ln\left(\frac{1}{\delta}\right) \quad \qedsymbol
\end{aligned}
\]
